\documentclass[a4paper]{article} %style de document
\usepackage[utf8]{inputenc} %encodage des caractères
\usepackage[french]{babel} %paquet de langue français
\usepackage[T1]{fontenc} %encodage de la police
\usepackage[top=2cm,bottom=2cm,left=2cm,right=2cm]{geometry} %marges
\usepackage{graphicx} %affichage des images
\usepackage{hyperref}%rend actif liens et références
\usepackage{verbatim}%permet insertion de texte brut (du style le logo Latex)
%\usepackage{amssymb} %collection de symboles
\usepackage[dvipsnames]{xcolor}%importe les couleurs, l'option permet d'avoir encore plus de couleurs
\usepackage{sectsty}%permet de changer les couleurs des sections/titres, etc
\usepackage{tikz}%pour faire des figures
\definecolor{astral}{RGB}{46,116,181}
\sectionfont{\color{red}}
\subsectionfont{\color{Blue}}
\subsubsectionfont{\color{NavyBlue}}
\usepackage{appendix} % Pour l'annexe


\title{
\LARGE{\em{Rapport Conception de Logiciels}}\\\vspace*{0.5cm}
UE Conception de Logiciels\\
Licence Informatique
}
\author{
Elie MALBEC - Alex LEFEVRE -  Yoann Kablan - Vouvou Brandon\\
Groupe 1B\\
Sujet : Optimisateur de Wargame
}

\date{2018 - 2019}

\begin{document} %début du document
\newpage
\newpage
	\null % Page blanche après page de garde
\newpage

\tableofcontents
\newpage

\section{Introduction}
Le choix du projet Ricochets Robots s'est fait par vote nous avons choisi ce projet a la suite d'un vote collectif des membres du groupe car aucun des sujets n'attirait tous les membres du groupe en même temps et comme la plupart des membres avaient plus ou moins envie .C'est donc ainsi que nous avons choisi ce projet.Ainsi il est nécessaire de se demander ce qu'est-ce que ricochet robot ? Ricochet Robots (Rasende Roboter pour la première édition en allemand) est un jeu de société créé par Alex Randolph et illustré par Franz VohwinkelLe jeu est composé d'un plateau, de tuiles représentant chacune une des cases du plateau, et de pions appelés  robots . La partie est décomposée en tours de jeu, un tour consistant à déplacer les robots sur un plateau n'an d'en amener un sur l'une des cases du plateau. Les robots se déplacent en ligne droite et avancent toujours jusqu'au premier mur qu'ils rencontrent. On peut aussi bien y jouer seul qu'à un grand nombre de participants.

\section{Objectifs du projet}	
	\subsection{Ce qu’il fallait faire}
L'objectif du projet  solveur de ricochet robots  est de réaliser un programme permettant de trouver une solution optimale pour toute situation du jeu sachant que le principe même du jeu est de trouver en moins d'une minute la séquence de mouvements qui permettra à un robot d'atteindre un objectif désigné sur une des cases du plateau de jeu.
	\subsection{Ce qui existe déjà}
(.......................)

\section{Fonctionnalités implémentées}

\subsection{DESCRIPTION  DES FONCTIONNALITÉS }
Au cours du projet , de multiples fonctionnalités ont été implémentées...Pour commencer nous avons implémenté le jeu et sa structure c'est-à-dire le plateau et tous les éléments qui le constitue ce plateau sont un plateau de n*m objets de types cases qui ont (…) . Ce plateau est constitué de plusieurs éléments c'est-à-dire les murs qui sont placés de façon aléatoire ou chargé à partir d’un fichier ces murs sont comme toute logique non traversable et formant souvent des angles qui contiennent souvent ou pas des cibles . Ensuite nous avons implémenté les différentes entités du jeu : les robots qui sont dotés de plusieurs attributs dont la position  et l’identifiant ils ont aussi la possibilité de se déplacer en ligne droite uniquement  ensuite les cibles qui ont les mêmes attributs que les robots qui sont entre autres la position et l’identifiant ainsi dans le plateau de jeu une de ses cibles est considéré comme  l'objectif principal . Nous avons accompagné cela d’une interface graphique là où il est possible d'afficher le plateau présent dans la console, enregistrer le plateau  ensuite  d’exécuter l’algorithme A* et enfin de remettre à zéro le plateau. Notre jeu ne se joue pas en console bien qu’il soit possible d’afficher le plateau de jeu avec des robots et même des déplacements dans la console.puis enfin de l’algorithme de résolution A* qui partant de la position de départ d’un robot trouve les positions intermédiaires jusqu’à la résolution du jeu . 

		\subsubsection{ ORGANISATION DU PROJET 
}
L'organisation du projet s'est déroulée en plusieurs étapes importantes, la première concerne la structure du projet. Les premières séances ont été consacrées au choix du projet. Après cela fallait commencer par coder le plateau et les éléments qui le constituent Bien avant d’avoir commencé à coder ensemble le jeu  Élie avait déjà une implémentation du plateau de jeu, ainsi Lors de la deuxième séance de Tp nous avons commencé à coder le plateau et les éléments qui le constituent c'est-à-dire les murs et les autres entités nous créons alors la première version de notre  jeu ricochets robot qui était la version robot 1.1 travail qui fut validé par tous les membres du groupe.Par la suite Alex a avancé sur la structuration du plateau  pendant que Yoann et Élie  commençaient à implémenter les déplacements des robots et les possibilités de déplacements des robots dans le plateau . Après plusieurs implémentions et test nous avons réuni nos idées pour finaliser cela. Une première version du projet était fonctionnelle en console et permettait déjà de :  choisir la cible principale de façon aléatoire, de sélectionner le robot principal, d’effectuer des déplacements, de charger un plateau prédéfini, de générer un plateau avec des constituants placés aléatoirement tout en respectant les normes du jeu.Après cette phase, Nous nous sommes directement penchés sur l’interface graphique partie où les connaissances d’Alex nous ont beaucoup aidés. À partir de ce moment , le projet s’est scindé en deux parties : Alex travaillait sur l’interface pendant que le reste du groupe travaillait sur l’implémentation de l’algorithme A* .

(………………………………….)

\section{ÉLÉMENTS  TECHNIQUE }	
\subsubsection{ALGORITHMES}

(………………………………….)
\subsubsection{STRUCTURES DE DONNÉES }

(………………………………….)
\subsubsection{BIBLIOTHÈQUES}

(………………………………….)

\section{ARCHITECTURES DU PROJET }
\subsubsection{DIAGRAMME DE CLASSES }

(………………………………….)
\subsubsection{ CAS D’UTILISATION}

(………………………………….)
\subsubsection{CHAÎNES DE TRAITEMENT }

(………………………………….)


\section{EXPÉRIMENTATIONS ET USAGES }
\subsubsection{CAPTURES D’ÉCRANS }


(………………………………….)
\subsubsection{ MESURES DE PERFORMANCES }

(………………………………….)

\section{CONCLUSION}
\subsubsection{RÉCAPITULATIF DES FONCTIONNALITÉS PRINCIPALES }

(………………………………….)
\subsubsection{PROPOSITIONS D’AMÉLIORATIONS }

(………………………………….)

\end{document} %fin du document

